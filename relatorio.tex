\documentclass{ufersa}

\begin{document}

\section{Descrição da Organização}
A organização é a Bee Delivery uma empresa que liga entregadores de varias 
modalidades a demandas de entregas de empresas, no mercado a 5 anos com um 
modelo de Startup .

\section{Descrição do ambiente tecnológico}
O ambiente de desenvolvimento é destinado à plataforma mobile, as práticas e experiências relatadas serão diretamente relacionadas a linguagem de programação javascript e a biblioteca React native, porém coexistindo com as linguagens php, com a biblioteca Laravel, SQL e noções sobre AWS onde se encontra a infraestrutura.

\section{Relatório das atividades desenvolvidas pelo aluno}
\subsection{Melhorias no código do aplicativo motoboy.}
\subsubsection{React Native}

	É um Framework que utiliza a linguagem Javascript baseada na biblioteca React, o React é um projeto open-source criado pelo Facebook, utilizada para criar interfaces de usuário, uma característica forte da biblioteca é a utilização de componentes que são elementos personalizáveis e reutilizáveis. O React native utiliza o React como base para criar aplicações moveis para as plataformas IOS e Android, muito utilizada por se tornar mais barata de desenvolver aplicações multiplataforma, 

\subsubsection{Flux}
Sensação de dados síncronos, evita renderizações em cascata.

\subsubsection{React redux}
\subsubsection{Styled components}
\subsubsection{React query}
\subsubsection{Native base}
\subsubsection{MVVM}
	É um uma estrutura de software de camadas semelhante as mais usadas em linguagem mais consolidadas. 

Todas as tecnologias citadas acima são utilizadas hoje no projeto em que vivencio, existe um processo de migração da estrutura antiga com flux e redux, para uma nova estrutura que utiliza react query, native base e mvvm.

\subsubsection{Conhecimentos adquiridos com outras equipes}

Às vezes é necessário formatar os modelos de dados no aplicativo para melhor desempenho da api, os modelos de dados do front-end não são os mesmos que uma api, ao entender isso deve conversar no modo mais eficiente para cada caso e estar disposto a mudar o que já funciona. 

\section{Reflexão por parte do aluno sobre as principais contribuições ao projeto}
\section{Relação dos principais conhecimentos obtidos nas disciplinas do curso e que foram de importância para o estágio}
\section{Reflexão sobre as dificuldades enfrentadas pelo estagiário na organização}
\section{Relação de tópicos que poderiam ser estudados no curso de Computação e que foram necessários no estágio.}
\end{document}
